\documentclass[a4paper,8pt]{article}
\usepackage{graphicx,caption}
\usepackage[bottom=1in,top=1in, left=1in, right=1in]{geometry}
\usepackage{subcaption}
\graphicspath{{figures/}}
\usepackage{geometry}
\pagenumbering{gobble}
\usepackage{sidecap}
\usepackage[export]{adjustbox}
\usepackage[margin=1cm]{caption}
\captionsetup[subfigure]{labelformat=empty}
\usepackage{pdfpages}

\usepackage{sectsty}

\sectionfont{\fontsize{12}{15}\selectfont}

%micro
\usepackage{siunitx}

%hyperlinks
\usepackage{hyperref}
\hypersetup{
    colorlinks=true,
    linkcolor=blue,
    filecolor=blue,      
    urlcolor=blue,
}

%top-right corner logo
\usepackage[T1]{fontenc}
\usepackage[ngerman]{babel}

%to add text in margins
\usepackage[absolute]{textpos}
\setlength{\TPHorizModule}{1mm}
\setlength{\TPVertModule}{1mm}

\usepackage{blindtext}

%nicer date at bottom:
\usepackage[yyyymmdd]{datetime}

\begin{document}
%date 
\renewcommand{\dateseparator}{--}

\newcommand\AtPageUpperRight[1]{\AtPageUpperLeft{%
   \makebox[\paperwidth][r]{#1}}}

\AddToShipoutPictureBG*{%
  \AtPageUpperRight{\raisebox{-\height}{\includegraphics[trim={0 0 10cm 0},clip, width=8cm]{Icos_cp_Logo_RGB}}}}

\newcommand\stationcode{TRN180}
\begin{flushleft}
\begin{huge}
 \textbf{\input{texts/\stationcode_text_1.txt}station characterization}
\end{huge}

\begin{large}
\bigskip
Station characterization based on STILT model footprints for 2018, an anthropogenic emissions database, a biogenic flux model and ancillary datalayers. More details are found at the end of this document.
\input{texts/\stationcode_text_1.txt}is \input{texts/\stationcode_text_2.txt}\input{texts/\stationcode_text_3.txt}located in \input{texts/\stationcode_text_4.txt}(latitude: \input{texts/\stationcode_text_5.txt}\unskip°N, longitude: \input{texts/\stationcode_text_6.txt}\unskip°E).
\end{large}
\end{flushleft}

\begin{figure}[!h]
\includegraphics[width=0.53\textwidth]{\stationcode_figure_1}
\raisebox{5.3cm}[0pt][0pt]{%
\hspace{-0.35cm}%
\captionsetup{labelformat=empty}
\parbox{7.9cm}{\caption{\begin{small}The \textbf{sensitivity area map} shows the average footprint/sensitivity area for 2018. The darker the colour the more important the area was as a potential source to the measured concentrations. The total sensitivity for the surface varies between stations and \protect\input{texts/\stationcode_text_1.txt}is in the \protect\input{texts/\stationcode_text_7.txt}of the ICOS certified atmospheric stations.\end{small}}}}
\end{figure}


\begin{figure}[!h]
\begin{subfigure}[t]{0.5\textwidth}
\includegraphics[width=0.85\linewidth]{\stationcode_figure_3}
\centering
\captionsetup{width=.8\linewidth}
\caption{\begin{small}The \textbf{population sensitivity map} is the result of the average sensitivity map multiplied by the number of people living within each footprint cell. Relative to other ICOS certified atmospheric stations, \input{texts/\stationcode_text_1.txt}is in the \input{texts/\stationcode_text_8.txt}when it comes to sensitivity to population.\end{small}}
\end{subfigure}%
\begin{subfigure}[t]{0.5\textwidth}
\includegraphics[width=0.85\linewidth]{\stationcode_figure_2}
\centering
\captionsetup{width=.8\linewidth}
\caption{\begin{small}The \textbf{point source contribution map} is the result of the average sensitivity map multiplied by the {\ensuremath{\mathrm{CO_2}}} emissions from point sources, like power plants and industrial facilities, within each footprint cell translated into expected influence on the {\ensuremath{\mathrm{CO_2}}} concentration at the station. Relative to other ICOS certified atmospheric stations, \input{texts/\stationcode_text_1.txt}is in the \input{texts/\stationcode_text_9.txt}when it comes to point source contribution.\end{small}}
\end{subfigure}
\end{figure}

%added- text in margin
\begin{textblock}{70}(115,280)
\noindent Date and time generated: \today \hspace{0.1cm} \currenttime
\end{textblock}

\pagebreak

\begin{figure}[!h]
\centering
\includegraphics[width=1\textwidth]{\stationcode_figure_6}
\end{figure}
%added- text in margin
\begin{textblock}{70}(115,280)
\noindent Date and time generated: \today \hspace{0.1cm} \currenttime
\end{textblock}

\begin{flushleft}
\begin{small}The first three variables in the \textbf{seasonal variations table} are the results of summarizing all the cells in the average footprint from December 2017 to December 2018 (to be able to break it down to the meteorological seasons). These values are found in the ``Annual'' column. Average footprints for the different parts of the year have in turn been computed, multiplied by the ancillary datalayers, and calculate relative (\%) to the average for the whole year. The remaining three variables – gross ecosystem exchange (GEE), respiration and anthropogenic contribution – are the modelled averages. 
\end{small}
\end{flushleft}

\begin{figure}[!h]
\includegraphics[width=1\textwidth]{\stationcode_figure_7}
\end{figure}

\begin{flushleft}
\begin{small}The land cover breakdown within \input{texts/\stationcode_text_1.txt}\unskip's  2018 average footprint is shown in the  \textbf{land cover bar graph}. The total for each land cover type is found in the legend and their relative occurence in the different directions of the stations (north-east, east, south-east etc.) are indicated by the graph. \end{small}
\end{flushleft}

\pagebreak

\begin{flushleft}
\begin{large}
\textbf{Advanced figures}\\
\bigskip
We advice careful reading of the explanations before attempting to understand the following figures. For further information and understanding, please read the specifications at the end of this documnet.
\end{large}
\end{flushleft}


\begin{figure}[!h]
\includegraphics[width=1\textwidth]{\stationcode_figure_4}
\raisebox{5cm}[0pt][0pt]{%
\hspace{9.05cm}%
\captionsetup{labelformat=empty}
\parbox{7.9cm}{\caption{\begin{small}The \textbf{land cover polar graph} shows the distribution of land cover types around the station located in the center. It indicates the directions (15 degrees bins) in which the land cover types are found within the average footprint. The area of a type in the graph corresponds its relative contribution with the highest contributing type located closest to the center. \end{small}}}}
\end{figure}

\begin{figure}[!h]
\includegraphics[width=0.57\textwidth]{\stationcode_figure_5}
\raisebox{4.8cm}[0pt][0pt]{%
\hspace{-0.3cm}%
\captionsetup{labelformat=empty}
\parbox{7.9cm}{\caption{\begin{small}Representative ICOS certified atmospheric stations are compared in this \textbf{multiple variable graph}. \protect\input{texts/\stationcode_text_1.txt}\unskip's values are shown with the black line and the points' placements on the y-axis are determined relative to the minimum (0\%) and maximum (100\%) values given all ICOS certified atmospheric stations. The variables are the same as those in the seasonal variations table. \end{small}}}}
\end{figure}

%added- text in margin
\begin{textblock}{70}(115,280)
\noindent Date and time generated: \today \hspace{0.1cm} \currenttime
\end{textblock}
\pagebreak


\input{texts/specifications.tex}

\end{document}
